\capitulo{1}{Introducción}


Hoy en día existen multitud de situaciones en las que se requiere un reparto de turnos o guardias entre un grupo de personas (por ejemplo personal sanitario en servicios hospitalarios, servicios de emergencias extrahospitalarias, bomberos, protección civil y muchos). En muchas de éstas ocasiones, el reparto de turnos planificado podría mejorarse fácilmente si se llevaran a cabo cambios bilaterales que a veces no se llegan a producir porque las partes implicadas desconocen que existe esa oportunidad de mejora, o porque intuyen que los costes de encontrar esa mejora y llegar a  hacerla posible son excesivamente elevados. Otras veces los cambios sí que pueden llegar a producirse pero tras un largo período de negociación.

Actualmente, tras un proceso de investigación inicial,  

\section{Motivacion}\label{Motivacion}

Explicar por que he elegido este trabajo y no otro es también describir cuáles son mis objetivos personales 

\section{Estructura de la memoria}\label{estructura-de-la-memoria}

\begin{itemize}
\tightlist
\item
  \textbf{Introducción:} Donde se explica la motivación al proyecto
\item
  \textbf{Objetivos del proyecto:} tanto a nivel académico, personal como los objetivos alcanzados.
\item
  \textbf{Conceptos teóricos:} breve explicación de los conceptos
  teóricos que he tenido que adquirid previamente para la realización del proyecto.
\item
  \textbf{Técnicas y herramientas:} software principal y metodologías empleadas durante el proyecto.
\item
  \textbf{Aspectos relevantes del desarrollo:} detalles sobre el la realización del proyecto.
\item
  \textbf{Trabajos relacionados:} otros aspectos del desarrollo web.
\item
  \textbf{Conclusiones y líneas de trabajo futuras:} conclusiones
  obtenidas tras la realización de este trabajo fin de grado y hacia donde va en el futuro.
\end{itemize}
