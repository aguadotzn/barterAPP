\apendice{Documentación de usuario}

\section{Introducción}\label{introduccion-usuario}
En este capítulo se detalla como un usuario puede comenzar a usa la aplicación 
deberemos diferenciar dos aspectos diferentes:

\begin{itemize}
\tightlist
\item
  Aplicación Web. 
\item
  Aplicación móvil. 
\end{itemize}

\section{Usuario Web}
Se describe en esta sección lo que puede hacer el usuario con la web.  Las páginas de las que está compuesta la web son:

\begin{tabular}{|c|c|c|c|c|c|}
  \hline
   home & about & login & register & calendarhome & help  \\
  \hline
\end{tabular}

\subsection{Landing page}
El entrar en la web el usuario ve la primera página de la misma o \emph{landing page}, dónde puede encontrar diferente información acerca de la aplicación, quiénes somos, que hacemos u otras cosas de utilidad.


\subsection{Registro}

El usuario debe logearse para acceder a la web, para ello debe pulsar el botón sign in

\imagen{usuarioregistro}{\footnotesize{Vista Web: Registro. Fuente: Elaboración propia.}}



\subsection{Login}

Una vez registrado para acceder a la web debe insertar su correo electrónico y contraseña

\imagen{usuariologin}{\footnotesize{Vista Web: Login. Fuente: Elaboración propia.}}

\subsection{Navegación}

\imagen{usuarionavbar1}{\footnotesize{Vista Web: Navbar Exterior. Fuente: Elaboración propia.}}


\imagen{usuarionavbar2{\footnotesize{Vista Web: registro. Interior: Elaboración propia.}}

\subsection{Añadir turnos al calendario}

\subsection{Intercambiar turno}

\begin{itemize}

\item Enviar petición

\item Aceptar/Rechazar petición

\item Confirmar petición

\end{itemize}

\subsection{Ayuda}



\section{Usuario Móvil}


\subsection{Registro}

\subsection{Login}

\subsection{Añadir turnos al calendario}

\subsection{Intercambiar turno}

\begin{itemize}

\item Enviar petición

\item Aceptar/Rechazar petición

\item Confirmar petición

\end{itemize}

\subsection{Ayuda}









