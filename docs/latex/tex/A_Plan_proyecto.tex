\apendice{Plan de Proyecto Software}

\section{Introducción}\label{introduccion-plan}
En este capitulo se detalla la planificación del proyecto. Como gestor de tareas se comenzó utilizando \emph{Trello+Github} pero más tarde se pasó a utilizar \emph{Zenhub,} extensión de Google Chrome que permiten integrar los \emph{boards} dentro del mismo repositorio de código alojado en \emph{github}. Ya se han dado más detalles en la memoria del proyecto.

Se ha utilizado metodologías ágiles para el desarrollo del proyecto y de este modo, se ha realizado un desarrollo dividido en iteraciones. Terminada una iteración empezaba la siguiente y se agregaban a las tareas planeadas las que no habían sido completado de la iteracción precedente. Las iteraciones del proyecto estaban pensadas para durar una semana aproximadamente. No obstante, hay alguna excepción en la que la iteración duró más tiempo.

La fase de planificación se puede dividir a su vez en:

\begin{itemize}
\tightlist
\item
  Planificación temporal.
\item
  Estudio de viabilidad.
\end{itemize} 

La primera parte me centro en la programación y desarrollo de la aplicación.Es decir elaboro un programa de tiempos con una serie de tareas a seguir para cumplimentar el proyecto.

La segunda parte se centra en el estudio de viabilidad. De la misma manera desde la segunda semana de marzo vengo realizando un plan de empresa con el programa Yuzz por lo que ello me va a facilitar el estudio de viabilidad de mi proyecto. Se desarrollará tanto la viabilidad legal como la económica. 


\section{Planificación temporal}\label{planificacion-temporal}
Desde inicio del proyecto se planteó utilizar una metodología ágil como
\emph{Scrum} para la gestión del proyecto. Aunque no se ha seguido al 100\% la
metodología al tratarse de un proyecto para la Universidad, sí que se ha aplicado
en líneas generales una filosofía ágil y metódica.

A continuación se describen los diferentes \emph{sprints} que se han
realizado. Dentro de \emph{github} cada \emph{milestone} recibe el número del sprint asignado y dentro de cada uno de ellos existen diferentes tareas que describiré a continuación


\subsection{Sprint 0: 09/01/17 - 15/01/17}\label{sprint0}

Tareas:

\begin{itemize}
	\item Primera tarea.
	\item Segunda tarea.
\end{itemize}

%\imagen{sprint0}{Detalle sprint 0}

Breve texto describiendo las tareas

\subsection{Sprint 1: }\label{sprint1}

Tareas:

\begin{itemize}
	\item Primera tarea.
	\item Segunda tarea.
\end{itemize}

%\imagen{sprint1}{Detalle sprint 1}

Breve texto describiendo las tareas

\subsection{Sprint 2: }\label{sprint2}

\subsection{Sprint 3: }\label{sprint3}

\subsection{Sprint 4: }\label{sprint4}

\subsection{Sprint 5: }\label{sprint5}

\subsection{Sprint 6: }\label{sprint6}

\subsection{Sprint 7: }\label{sprint6}


\section{Estudio de viabilidad}\label{estudio-viabilidad}
En esta sección se lleva a cabo un estudio para comprobar la viabilidad del proyecto realizado. Paralelamente al desarrollo de la aplicación, como ya se ha nombrado en la memoria, el proyecto formó parte del programa YUZZ para jóvenes emprendedores en el que durante cinco meses realicé un plan de empresa completo. Se detalla por tanto en un documento que adjuntaré al proyecto un estudio de viabilidad exhaustivo  y muy completo en el que se incluyen entre otras cosas: plan de marketing, plan de financiación, estudio de viabilidad o plan de puesta en marcha del negocio a cinco años vista. 

Por lo tanto en esta sección voy a realizar un resumen del documento descrito en el párrafo anterior en el que como conclusión lograremos afirmar si el proyecto es o no rentable económicamente hablando.

\subsection{Viabilidad económica}\label{viabilidad-economica}

La viabilidad económica 

\subsubsection{Análisis de costes}\label{costes}
Económica

\subsection{Viabilidad legal}\label{viabilidad-legal}
La viabilidad legal se centra principalmente en el estudio de las licencias software utilizadas y en la licencia que se le va a ser asignada a las diferentes aplicaciones desarrolladas.


