\apendice{Plan de Proyecto Software}

\section{Introducción}\label{introduccion-plan}
En este capitulo se detalla la planificación del proyecto. Como gestor de tareas se comenzó utilizando \emph{Trello+Github} pero más tarde se pasó a utilizar \emph{Zenhub,} extensión de Google Chrome que permite integrar los \emph{boards} dentro del mismo repositorio de código alojado en \emph{github}. 

Se han utilizado metodologías ágiles para el desarrollo del proyecto y de este modo, se ha realizado un desarrollo dividido en iteraciones. Terminada una iteración empezaba la siguiente y se agregaban a las tareas planeadas las que no habían sido completado de la iteracción precedente. Las iteraciones del proyecto estaban pensadas para durar unos diez días aproximadamente. No obstante, hay alguna excepción en la que la iteración duró más tiempo. También existe alguna demora entre algún sprint debido a que tenía demasiada carga de trabajo de las asignaturas, trabajaba o estaba de viaje. 

Este primer apéndice se puede dividir a su vez en:

\begin{itemize}
\tightlist
\item
  Planificación temporal.
\item
  Estudio de viabilidad.
\end{itemize} 

La primera parte me centro en la programación y desarrollo de la aplicación. Es decir elaboro un programa de tiempos con una serie de tareas a seguir para cumplimentar el proyecto.

La segunda parte se centra en el estudio de viabilidad. Es necesario destacar en este punto que desde la segunda semana de marzo vengo realizando un plan de empresa con el \emph{programa Yuzz} por lo que ello me va a facilitar el estudio de viabilidad de mi proyecto. Se desarrollará tanto la viabilidad legal como también la económica. 


\section{Planificación temporal}\label{planificacion-temporal}
Desde el comienzo del proyecto se planteó utilizar una metodología ágil como
\emph{Scrum} para la gestión del proyecto. Aunque no se ha seguido al 100\% la
metodología al tratarse de un proyecto para la universidad, sí que se ha aplicado
en líneas generales una filosofía ágil y metódica. La diferencia fundamental radica en que esta metodología esta pensada para equipos y no para individuos.

A continuación se describen los diferentes \emph{sprints} que se han
realizado. Dentro de \emph{github} cada \emph{milestone} recibe el número del sprint asignado y dentro de cada uno de ellos existen diferentes tareas que describiré a continuación. A cada tarea le acompaña un número a la derecha el cuál es denominado \emph{story point}, de alguna manera sirve para realizar una estimación de lo que te va llevar completar esa determinada tarea. En mi caso concreto el \textbf{1} resulta ser el más bajo lo cuál indicaría que no más de dos o tres horas con cada tarea y el  \textbf{13} el más alto lo cuál significa varios días de trabajo.


\subsection{Sprint 0: 18/02/2017 - 28/02/2017}\label{sprint0}

Tareas principales:

\begin{itemize}
	\item Terminar formación.
	\item Aprender \emph{Sonarqube}.
	\item Inicio Back-End.
    \item Inicio Decidir base de datos a emplear.
\end{itemize}

\imagen{sprint0}{\footnotesize{Detalle sprint 0.}}

La primera vez que hice uso de \emph{ZenHub} ya llevaba algún tiempo formándome, de ahí el primer \emph{Issue} del \underline{Sprint 0}. Esta primera toma de contacto fue para comenzar a desarrollar el Back-End de la aplicación, además se consultaron varias fuentes para decidir el tipo de base de datos emplear.

\subsection{Sprint 1: 18/02/2017 - 28/02/2017}\label{sprint1}

Tareas principales:

\begin{itemize}
	\item Correción de errores en Back-End.
	\item Desarrollo Back-end.
	\item Errores en base de datos.
\end{itemize}

\imagen{sprint1}{\footnotesize{Detalle sprint 1.}}

El  \underline{Sprint 1} sirvió para continuar con el Back-end de la aplicación, sin dudarlo fue un de las partes más complicadas al pelearme con bases de datos con conceptos nuevos por lo que tuve numerosos bugs a la hora de guardar los usuarios en la base de datos.

\subsection{Sprint 2: 15/03/2017 - 31/03/2017}\label{sprint2}

Tareas principales:

\begin{itemize}
	\item Corrección de errores en Back-End.
	\item Inicio Front-End.
	\item Bug en base de datos.
	\item Comienzo a leer sobre la documentación.
\end{itemize}

\imagen{sprint2}{\footnotesize{Detalle sprint 2.}}

El \underline{Sprint 2} fue el momento donde una vez tenía un back-end sólido debía trasladarlo a la parte del usuario por lo que comencé a realizar la parte del front-end.

\subsection{Sprint 3: 07/04/2017 - 15/04/2017}\label{sprint3}

Tareas principales:

\begin{itemize}
	\item Subir documentación.
	\item Elección del calendario.
	\item Corrección en componentes.
\end{itemize}

\imagen{sprint3}{\footnotesize{Detalle sprint 3.}}

El \underline{Sprint 3} se centra en la parte del calendario sobre todo, además de algo de documentación y corregir los errores que he arrastrado del back-end.

\subsection{Sprint 4: 16/04/2017 - 22/04/2017}\label{sprint4}

Tareas principales:

\begin{itemize}
	\item Actualizar a Angular CLI.
	\item Heroku y MLab
	\item Documentación
\end{itemize}

\imagen{sprint4}{\footnotesize{Detalle sprint 4.}}

El \underline{Sprint 4} es más corto dado que requiere un menor tiempo en realizar las tareas.

\subsection{Sprint 5: 30/04/2017 - 07/05/2017}\label{sprint5}

Tareas principales:

\begin{itemize}
	\item Angular CLI.
	\item Bugs: base de datos 
	\item Bugs: calendario
\end{itemize}

\imagen{sprint5}{\footnotesize{Detalle sprint 5.}}

El \underline{Sprint 5} fue complicado debido a que cambiar a Angular CLI resulta más sencillo a la hora de desplegar en servidor pero hay que saber cómo funciona realmente los proyectos en Angular CLI.

\subsection{Sprint 6: 09/05/2017 - 16/05/2017}\label{sprint6}

Tareas principales:

\begin{itemize}
	
		\item Error en turnos diarios, se decide implementar turnos 24 horas para comprobar si el algoritmo intercambia los turnos.
	\item Anexos en la documentación.
\end{itemize}

\imagen{sprint6}{\footnotesize{Detalle sprint 6.}}

El \underline{Sprint 6} se centra en que el algoritmo intercambie los turnos de manera eficiente. 

\subsection{Sprint 7: 24/05/2017 - 31/05/2017}\label{sprint7}

Tareas principales:

\begin{itemize}

		\item Tras consultar a Luis he obviado un importante elemento y es necesario refactorizar el algoritmo.
	\item Se incluye una parte para tests pero no me da tiempo.
\end{itemize}

\imagen{sprint7}{\footnotesize{Detalle sprint 7.}}

El \underline{Sprint 7} se centra en en conseguir que el algoritmo funcione para todos los tipos de turnos.

\subsection{Sprint 8: 01/06/2017 - 10/06/2017}\label{sprint8}

Tareas principales:

\begin{itemize}
	\item Completar últimos componentes
	\item Terminar documentación.
	\item Bugs: Travis. 
	\item Bugs: Heroku y fallo en el algoritmo.
	\item Pruebas de calidad del código (Como \emph{SonarQube} no da soporte a Typescript decidí prescindir de esta herramienta y utilizar otras.)
\end{itemize}

\imagen{sprint8}{\footnotesize{Detalle sprint 8.}}

El \underline{Sprint 8} se centra en corregir errores, uno de los cuales no consigo solventar, y realizar pruebas en la calidad del código. 

\subsection{Sprint 9: 10/06/2017 - 20/06/2017}\label{sprint9}

Tareas principales:

\begin{itemize}
	\item Refactorización.
	\item App móvil: scafolding básico y creación de componentes.
	\item Envío a Luis de la documentación para posibles comentarios.
	\item Error en el servidor.
	
\end{itemize}

\imagen{sprint9}{\footnotesize{Detalle sprint 9.}}

El \underline{Sprint 9} trata de refactorizar la app para que la calidad del código aumente. Se consigue en un tanto por ciento considerable. 

\subsection{Sprint 10: 20/06/2017 - 30/06/2017}\label{sprint10}

Tareas principales:

\begin{itemize}
	\item 
\end{itemize}

%\imagen{sprint10}{\footnotesize{Detalle sprint 10.}}

El \underline{Sprint 10} 

\subsection{Sprint 11: 30/06/2017 - 10/07/2017}\label{sprint11}

Tareas principales:

\begin{itemize}
	\item 
\end{itemize}

%\imagen{sprint10}{\footnotesize{Detalle sprint 11.}}

El \underline{Sprint 11} 


\section{Estudio de viabilidad}\label{estudio-viabilidad}
En esta sección se lleva a cabo un estudio para comprobar la viabilidad del proyecto realizado. Paralelamente al desarrollo de la aplicación, como ya se ha nombrado anteriormente, el proyecto formó parte del \emph{programa YUZZ } para jóvenes emprendedores en el que durante cinco meses realicé un plan de empresa completo. Se detalla por tanto en un documento extra que adjuntaré un estudio de viabilidad exhaustivo  y muy completo en el que se incluyen entre otras cosas: plan de marketing, plan de financiación, estudio de viabilidad o plan de puesta en marcha del negocio a cinco años vista. Así como también encuestas, logotipado o un \emph{landing page} necesaria para la presentación del proyecto ante el jurado. 

Por lo tanto en esta sección voy a realizar un resumen del documento descrito en el párrafo anterior en el que como conclusión definitiva tendremos un boceto de lo que supondría transformar un  trabajo fin de grado en un producto real. Así mismo voy a intentar adaptarlo a las condiciones que se exigen en el proyecto dado que el plan de empresa adjunto como extra es un estudio de viabilidad íntegro de aquí a cinco años por lo que resulta ser más extenso y detallado. Se intentará por tanto aquí realizar una estimación.

\subsection{Viabilidad económica}\label{viabilidad-economica}

La viabilidad económica es la parte donde lograremos detectar si el proyecto es o no rentable económicamente hablando.

\subsubsection{Análisis de costes}\label{costes}

Lo subdividiremos en diferentes tipos de costes.

\begin{description}
	\item[Coste de personal] Se considerará que el proyecto ha sido desarrollado en un periodo de cinco meses.  Considerando que se ha trabajado unas 6 horas al día cada semana, y que el programador, que en este caso es una sola persona, ha percibido un salario de 13 \euro /hora, el coste del personal por lo tanto se resume en la siguiente tabla:
	
\begin{table}[htbp]
\begin{center}
\begin{tabular}{|l|l|}
\hline
 & Total \\
\hline \hline
13 \euro /hora * 6 horas/día &   78 \euro /día \\ \hline
78 \euro /día * 5 dias/semana &   390 \euro /semana \\ \hline
390 \euro /semana * 4 semanas/mes &   1560 \euro /mes \\ \hline
\textbf{Coste total salario} &   7800 \euro /5 meses \\ \hline
\end{tabular}
\caption{Tabla salarios.}
\label{tabla:salarios}
\end{center}
\end{table}
	
	\item[Coste de seguridad social] al coste de personal hay que sumar el coste de seguridad social. 
	
	
	\begin{table}[htbp]
\begin{center}
\begin{tabular}{|l|l|}
\hline
 \emph{Seguridad social} &  \\
\hline \hline
 & \emph{23,60} \% en contigencias comunes \\ \hline
&   \emph{7,70} \% en concepto de desempleo \\ \hline
 &   \emph{0,10} \% en concepto de formación personal. \\ \hline
\textbf{Gastos Seguridad Social}  &   7800 \euro  * 31,40 \% =  2449,20 \euro \\ \hline
\textbf{Gastos personal}  &  7800 \euro  + 2449,20 \euro  = 10249,2 \euro  \\ \hline
\end{tabular}
\caption{Tabla seguridad social.}
\label{tabla:ssocial}
\end{center}
\end{table}
	
	\item[Coste de software] Todo el software utilizado para este proyecto es gratuito y de libre distribución. 
	
	\item[Coste de Hardware] En este apartado se analizan los costes derivados de la compra de los materiales que son necesarios para llevar a cabo el proyecto. Según la ley de Impuesto de Sociedades, la duración media del inmovilizado  hardware oscila entre 8 y 3 a\~nos. Yo consideraré 3 años. Más detalles en la siguiente tabla.  
	
\begin{table}[htbp]
\begin{center}
\begin{tabular}{|l|l|}
\hline
 \emph{Hardware utilizado}&  \\
\hline \hline
 Portátil MacBook Pro 15'' & 1650 \euro  \\ \hline
Monitor externo Samsung 22'' &    107'99 \euro\\ \hline
\textbf{Gastos Hardware}  &   1650 \euro + 107,99  =  1757,99 \euro \\ \hline
\textbf{Coste amortización}  &  1757,99 \euro  / 36 meses = 48,8 \euro / mes  \\ \hline
\end{tabular}
\caption{Tabla costes hardware.}
\label{tabla:hardware}
\end{center}
\end{table}

\begin{table}[htbp]
\begin{center}
\begin{tabular}{|l|l|}
\hline
\textbf{Coste final hardware} (5 meses)  &  48,8 \euro / mes  *  5 meses = 244 \euro   \\ \hline
\end{tabular}
\caption{Coste final hardware.}
\label{tabla:costefinalhardware}
\end{center}
\end{table}


\subsection{Coste final desarrollo del proyecto}\label{coste-final}

El coste final del desarrollo del proyecto, con una duración estimada de cinco meses es el siguiente.  

\begin{table}[htb]
\begin{center}
\begin{tabular}{|l|l|}
\hline
\textbf{Coste total final del desarrollo proyecto}  &  244 \euro +  10249,2 \euro = 10493 \euro   \\ \hline
\end{tabular}
\caption{Coste proyecto final..}
\label{tabla:costefinal}
\end{center}
\end{table}

\end{description}

\subsection{Viabilidad legal}\label{viabilidad-legal}
La viabilidad legal se centra principalmente en el estudio de las licencias software utilizadas. Realizaremos una tabla resumen sobre las partes más importantes de la aplicación y la licencia que emplean. Realmente son muchos los módulos que se han empleado para el desarrollo pero la mayoría tienen la misma licencia por lo que no he nombrado todos simplemente los que más importantes parecen. 

\begin{table}[H]
\begin{center}
\begin{tabular}{|l|l|}
\hline
Dependencias & Licencia \\
\hline \hline
\emph{angular-calendar} & \href{https://opensource.org/licenses/MIT}{MIT}\\ \hline
\emph{@angular/cli} & \href{https://opensource.org/licenses/MIT}{MIT}\\ \hline
\emph{@types/node} & \href{https://opensource.org/licenses/MIT}{MIT}\\ \hline
\emph{primeng} & \href{https://opensource.org/licenses/MIT}{MIT}\\ \hline
\emph{tslint} & \href{https://opensource.org/licenses/Apache-2.0}{Apache-2.0}\\ \hline
\emph{typescript} & \href{https://opensource.org/licenses/Apache-2.0}{Apache-2.0}\\ \hline
\emph{font-awesome} & \href{https://opensource.org/licenses/OFL-1.1}{OFL-1.1}\\ \hline
\emph{mongoose} & \href{https://opensource.org/licenses/MIT}{MIT}\\ \hline
\emph{express} & \href{https://opensource.org/licenses/MIT}{MIT}\\ \hline
\emph{bootstrap} & \href{https://opensource.org/licenses/MIT}{MIT}\\ \hline

\end{tabular}
\caption{Tabla resumen-licencias.}
\label{tabla:licencias}
\end{center}
\end{table}



Una licencia software es un contrato entre el autor o titular de los derechos de explotación o distribución y el usuario consumidor, usuario profesional o empresa, para la utilización del software cumpliendo una serie de términos y condiciones establecidas dentro de sus cláusulas. Todo el software que empleo es su amplia mayoría es libre ya que la mayoría de código viene o bien de desarrolladores \emph{amateur} o es libre desde su creación. Emplean por tanto las licencias descritas en la tabla anterior por lo que diferencia entre ellas varían en términos como el nombramiento del autor o la garantía , son licencia comunes en software libre \cite{githublicense} , además github provee una página para ayudarte en la elección de tu licencia \citep{githubchoose}. La menos permisiva de entre las descritas es \emph{MIT} \citep{mit}  y la más es la licencia del  \emph{Apache-2.0} \cite{apache}.


\imagen{licencias}{\footnotesize{Gráfico licencias Open Source in GitHub. Fuente:  \url{https://cartograf.net}.}}
cartograf.net

Tal y como he nombrado anteriormente, y el tutor así lo deseaba, el software puede ser modificado, ayudando así a que crezca en un futuro
por lo que he elegido \emph{Apache-2.0}. 


Para la documentación he elegido una licencia Creative commons, en concreto se ha elegido la \emph{Attribution-NonCommercial 4.0 International (CC BY-NC 4.0)} visible en \ref{commons} .


\imagen{creativecommons}{\footnotesize{Licencia Creative Commons.}}\label{commons}



