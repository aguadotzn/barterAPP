\apendice{Plan de Proyecto Software}

\section{Introducción}\label{introduccion-plan}
En este capitulo se detalla la planificación del proyecto. Como gestor de tareas se comenzó utilizando Trello+github pero más tarde se pasó a utilizar Zenhub, extensión de Chrome que permiten integrar los boards dentro de github. Una opción sin duda muchísimo mas cómoda. 
Se ha utilizado metodologías agiles para el desarrollo del proyecto y de este modo, se ha realizado un desarrollo dividido en iteraciones. Terminada una iteración empezaba la siguiente y se agregaban a las tareas planeadas las que no habían sido completado de la iteración precedente. Las iteraciones del proyecto estaban pensadas para durar una semana. No obstante, hay alguna excepción en la que la iteración duró dos semanas. 


\section{Planificación temporal}\label{planificacion-temporal}
En esta sección se hace un brevísimo resumen de las tareas que se han llevado a cabo dentro de cada interación
\subsection{Sprint 1}
\subsection{Sprint 2}
\subsection{Sprint 3}
\subsection{Sprint 4}
\subsection{Sprint 5}


\section{Estudio de viabilidad}\label{estudio-viabilidad}
Perfectamente detallado en el informe realizado para el YUZZ: plan de empresa de 102 páginas, no ya solo con la viabilidad de la herramienta sino con plan económico de aquí a cinco años vistas 


\subsection{Viabilidad económica}\label{viabilidad-economica}
Economica

\subsection{Viabilidad legal}\label{viabilidad-legal}
Legal


