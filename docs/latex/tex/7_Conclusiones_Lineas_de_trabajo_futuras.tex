\capitulo{7}{Conclusiones y Líneas de trabajo futuras}

En está última sección se exponen las conclusiones derivadas del trabajo. De la misma manera abordaremos las línea de trabajo futura que puede llevar la aplicación y que podrían ayudar a mejorarla y a posicionarla en el mercado si realmente se decide lanzarla.


  \section{Conclusiones proyecto}\label{conclusiones_proyecto}
  
\begin{itemize}
	\item Aplicación web en el servidor funcionando.
	\item Aplicación móvil instalada funcionando.
	\item Algoritmo óptimo.
\end{itemize}
  
  \section{Conclusiones personales}\label{conclusiones_personales}
	Me gustaría exponer también unas conclusiones personales a modo de lista sobre lo que este trabajo ha supuesto para mi.  
    
\begin{itemize}
	\item Sin duda es la parte del grado que más he aprendido por que he tenido que aprender, estudiar y descubrir casi todas las cosas utilizadas, la gran mayoría eran nuevas para mi por lo que ha sido un período muy fructífero desde el punto de vista académico
	\item El algoritmo me ha costado mucho más de lo esperado eso ha hecho que se demorarán otras partes del proyecto, que quizás no sean tan importantes pero un proyecto es un conjunto y no debería perder tanto tiempo en determinadas partes por que eso me resta tiempo para otras. Conclusión: aprender a planificarse mejor.
	\item Otras autocrítica es la siguiente: cuando se decide usar una herramienta establecer ese uso como determinado y final ya que sino se pierde un tiempo muy valioso en aprender, adaptarse y manejar correctamente otras herramientas. Si bien es cierto que esto es bueno por la parte de conocimientos, no resulta nada bueno si en un entorno real de trabajo se tienen unas fechas de entrega a cumplimentar.
	\item Realmente la solución aportada al problema real me gusta, me he involucrado con ella y he defendido la idea delante de un jurado aquí en España o en dos universidades del Reino Unido, a las cuales les entusiasmó la solución aportada aunque con algunos matices. Por lo que sin lugar a dudas me gustaría seguir con el proyecto, mejorándolo y haciendo un producto serio para el usuario.
	\item No es necesario abarcar tanto, es mejor hacer poco y bien que mucho y no tan bien.
\end{itemize}
  
   \section{Líneas de trabajo}\label{lineas_futuras}
   Es importante remarcar las línea de futuro que el proyecto pretende seguir.
  
\begin{itemize}
	\item Refactorizar parte móvil: sin duda requiere un refactorización ya que las prisas hacen que haya elementos extra. 
	\item Mejorar interfaz de usuario, tanto móvil como web, quizás actualmente no sean las mejores. Darle un lavado de cara a la interfaz de usuario puede ser interesante de cara a que la aplicación adquiere más seriedad.
	\item Incorporar un panel de usuario: para que el usuario se sienta cómodo y sea capaz de editar y eliminar sus propios datos, además de personalizar esta clase con una fotografía de ésta manera puede ser más sencillo de cara a identificar al compañero con el que vamos a intercambiar el turno.
	\item Incorporar un chat en tiempo real para que la iteracción entre los usuarios sea desde la propia plataforma interna y no dependa de ninguna otra. Los cambios de esta manera se hacen de una manera más directa.
\end{itemize}