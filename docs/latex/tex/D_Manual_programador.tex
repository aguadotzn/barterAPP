\apendice{Documentación técnica de programación}

\section{Introducción}\label{introduccion-programador}
En este capítulo vamos a adentrarnos en los detalles para tener un entorno con el que programar de la manera “más real” posible y programar así nuevas funcionalidades para barterAPP.

\section{Estructura de directorios}\label{directorios}

\section{Manual del programador}
En esta sección hay que tener en cuenta que el autor de este trabajo fin de grado a escogido unas herramientas, tanto para desplegar la app, como la base de datos como para desarrollar la aplicación pero que de ninguna manera resultan ser ni las únicas ni las mejores simplemente son unas herramientas que ha considerado utilizar pero existen muchas más que no son ni peores ni peores. 

Para desplegar la app hemos elegido heroku, para la base de datos mlab ( que dentro tiene servidores AWS, Google Cloud o Azure)

Pasos para montar en tu propio ordenador y desarrollar tu propia API (tener en cuenta que es software desde el que se realiza es un MACBOOK PRO, por lo que pueden existir cambios respecto a otros sitemas operativos. Trataré sin embargo ajustarme y dar detalles para instalarlo en cualquier entorno.

\subsection{Compilación, instalación y ejecución del proyecto}
Voy a tratar de explicar un desarrollo completo desde la instalación en local hasta la carga en un servidor. 
\subsection{Modo desarrollador / Modo produccion}

\subsection{Base de datos}
\subsection{Avisos}


