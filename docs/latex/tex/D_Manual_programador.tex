\apendice{Documentación técnica de programación}

\section{Introducción}\label{introduccion-programador}
En este capítulo vamos a dividirlo en dos partes, por un lado los condiciones específicas para la aplicación y por otro vamos a aprender de manera breve y sencilla a crear una aplicación gracias a \emph{Angular CLI}, la interfaz de línea de comandos de Angular. 

El por qué he decidido incluir la parte de \emph{Angular CLI} es por que considero que va aclarar muchas dudas acerca de cómo funciona esta tecnología y también para intentar comprender mejor como he realizado el proyecto. En diversas ocasiones para aprender a usar una tecnología tan solo hace falta tiempo pero en otras muchas el tiempo es limitado por lo que quizás con esta pequeña guía podemos comenzar a utilizar un framework cliente en muy pocos pasos y de una manera sencilla. Si bien es cierto que yo he necesitado mucho tiempo para comprender su funcionamiento y no lo he entendido gracias a la interfaz de comandos, ésta te permite agilizar los trámites de creación en un tanto por ciento considerable.

Este anexo tiene como objetivo analizar y documentar las necesidades funcionales que deberán ser soportadas por el sistema a desarrollar, es decir, en qué condiciones ha sido desarrollado, en qué condiciones se debe usar y cuáles son los requerimientos mínimos para que funcione.

\section{Requerimientos mínimos necesarios}\label{angularCLI}


\section{AngularCLI}\label{angularCLI}
Sabiendo los requisitos mínimos que son necesarios para proceder 

\subsection{¿Qué es typescript?}\label{typesript}
Esto se ha nombrado antes pero resulta esencial para entender el código de Angular

\begin{center}
    \begin{minipage}{0.9\linewidth}
        \vspace{5pt}%margen superior de minipage
        {\small
            \emph{TypeScript is a typed superset of JavaScript that compiles to plain JavaScript. Any browser. Any host. Any OS. Open source.}
        }
        \begin{flushright}
            (\cite{TypeScript Web})
        \end{flushright}
        \vspace{5pt}%margen inferior de la minipage
    \end{minipage}
\end{center}

\subsection{Comandos básicos}\label{cbasicos}

\subsection{Puesta en marcha}\label{typesript}


\section{Manual del programador}
En esta sección hay que tener en cuenta que el autor de este trabajo fin de grado a escogido unas herramientas, tanto para desplegar la app, como la base de datos como para desarrollar la aplicación pero que de ninguna manera resultan ser ni las únicas ni las mejores simplemente son unas herramientas que ha considerado utilizar pero existen muchas más que no son ni peores ni peores. 

Para desplegar la app hemos elegido heroku, para la base de datos mlab ( que dentro tiene servidores AWS, Google Cloud o Azure)

Pasos para montar en tu propio ordenador y desarrollar tu propia API (tener en cuenta que es software desde el que se realiza es un MACBOOK PRO, por lo que pueden existir cambios respecto a otros sitemas operativos. Trataré sin embargo ajustarme y dar detalles para instalarlo en cualquier entorno.

\subsection{Compilación, instalación y ejecución del proyecto}
Voy a tratar de explicar un desarrollo completo desde la instalación en local hasta la carga en un servidor. 


\subsection{Modo desarrollador }

\subsection{Modo producción}

\subsection{Estructura de directorios del proyecto}

\subsection{Base de datos}




\subsection{Avisos}


